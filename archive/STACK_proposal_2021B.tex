%%%%%%%%%%%%%%%%%%%%%%%%%%%%%%%%%%%%%%%%%%%%%%%%%%%%%%%%%%%%%%%%%%%
% This is a simple template for writing proposals for Keck
% telescope time through the CAS STACK (Swinburne Time
% Allocation Committee for Keck).
%
%    Version 2021B.v1
%
% Contributors: Themiya Nanayakkara, Tiantian Yuan, Jeff Cooke, Deanne Fisher, Michael Murphy.
% 
% To produce a PDF from this file execute the following from a
% command line:
%
% latex STACK_proposal.tex ; latex STACK_proposal.tex ;
%   dvips STACK_proposal.dvi -o STACK_proposal.ps ;
%   ps2pdf14 -sPAPERSIZE=a4 STACK_proposal.ps STACK_proposal.pdf
% 
% or, if you prefer,
%
% pdflatex STACK_proposal.tex ; pdflatex STACK_proposal.tex
%

% *!!!!INSTRUCTIONS!!!!*

% - A proposal consists of a Keck cover sheet and this proposal form. Follow instructions 
% from the CfP website to fill out the cover sheet and attach to the pdf output of
% this proposal. 

% - This proposal form describes the requirements and page limits for
% the various sections. Read instructions associated with each section
% and follow the guidelines. 
% Page limits:
% 	*  3-page maximum for the Scientific Justification section (including texts, figures and references).
% 	*  1-page maximum in total for the Technical Feasibility and Flexibility & Minimum Time sections.
%   *  0.5-page maximum for the JUSTIFICATION FOR  DIRECTOR'S DISCRETIONARY TIME (if applicable). 
% 	*  2-page maximum in total for the sections BACKUP PROGRAM, CONTRACT END, APPLICANTS' ROLES, 
% 	   STUDENT THESIS INVOLVEMENT, OTHER APPLICATIONS, FUTURE APPLICATIONS, and RESPONSE TO STACK FEEDBACK.
% 	* Target lists may be attached to the proposal as a separate sheet(s) with no page limit.

% - Add your previous Keck allocation and outcome in this database here: 
%   http://astronomy.swin.edu.au/research/keck_pubs/table.php (User authentication: credentials as for CAS WIKI).
%   The STACK expects published data for completed projects 1.5 years after the final observation date.
%   The lack of publications for completed projects will be considered in the ranking of the PIs proposals.
%   The STACK will only use this online database for your proposal ranking, so update your entry within 3 working days of your proposal submission.   

% Direct any questions to KeckTS@swin.edu.au

%  DO NOT EDIT OR REMOVE commands with comments which tell you not to!


%%%%%%%%%%%%%%%%%%%%%%%%%%%%%%%%%%%%%%%%%%%%%%%%%%%%%%%%%%%%%%%%%%%

\documentclass[12pt,a4paper]{article} % - DO NOT EDIT OR REMOVE!
\usepackage{fancyhdr,txfonts}         % - DO NOT EDIT OR REMOVE!

% Include your own packages here.  For example, if you want use the
% natbib package, leave the first two commands below uncommented
% (the second command specifies punctuation within citations; the
% example shown gives something like the MNRAS style).  The third
% command below (commented out) includes that commonly used to
% include figures. The fourth command might be useful while you're 
% drafting your proposal.

\usepackage{natbib}
\bibpunct{(}{)}{;}{a}{}{,}
\usepackage{graphicx}

% Enter the Keck cover sheet code, PI surname and proposal title
% below.  Please enter the Keck cover sheet code in the same format
% as given on the Keck cover sheet.  NOTE: Include any formatting
% commands (e.g. boldface, \boldmath commands etc.) for your
% proposal title here.

\newcommand\keckcode{2021B\_WXXXX} % Modify to your specific 
                                    % proposal ID number
\newcommand\PIsurname{PI SURNAME}
\newcommand\proposaltitle{Title of proposal} 

% Page layout - DO NOT EDIT OR REMOVE BETWEEN THE LINES!
%------------------------------------------------------------------
\topmargin=-20mm
\headheight=0mm
\textwidth=176mm
\textheight=266mm
\oddsidemargin=-8mm
\evensidemargin=-8mm
\renewcommand\baselinestretch{1.0}
\pagestyle{fancyplain}
\lhead{\fancyplain{}{{\bf\MakeUppercase{\keckcode}}}}
\rhead{\fancyplain{}{\bf\thepage}}
\chead{\fancyplain{}{{\bf\MakeUppercase{\PIsurname}}}}
\cfoot{}
%------------------------------------------------------------------
% Page layout - YOU MAY EDIT THESE IF YOU WISH.
%
% NOTE THAT IT IS STRONGLY RECOMMENDED NOT TO CROWD YOUR TEXT AND
% FORMATTING TOO MUCH BECAUSE IT MAKES READING THE PROPOSAL 
% DIFFICULT.  USE SIMPLE FORMATTING COMMANDS LIKE \smallskip, 
% \bigskip, \vspace ETC.

\parindent=1.5em % Paragraph indentation length (in units of the size
                 % of an "M", or change to cm, etc.).  If you want to 
                 % keep this for general usage but want a specific 
                 % paragraph not to be indented, use the \noindent 
                 % command at the start of that paragraph
\parskip=0.5em   % Space between paragraphs.

 \newenvironment{nscenter}
 {\parskip=0pt\par\nopagebreak\centering}
 {\par\noindent\ignorespacesafterend}




\begin{document}

% Proposal title - DO NOT EDIT OR REMOVE!
%\centerline{\large{\bf \proposaltitle}}\smallskip

\begin{nscenter}
{\large{\bf \proposaltitle}}
\end{nscenter}


\noindent{\bf SCIENTIFIC JUSTIFICATION:}

% {\it Instructions:}  
% You may use subheadings of your own choosing to make your text 
% clearer.
% References may be included in-line (e.g.~Salpeter, 1955, ApJ, 121,
% 161) and/or all together in a separate section (maybe under a heading
% "References" like that below) with only citations in the text
% \cite[e.g.][]{SalpeterE_55a}.  

% Example reference list.
% \renewcommand\refname{\normalsize REFERENCES:}
% \begin{thebibliography}{}\vspace{-1.5em} 
% \itemsep 0.0em 
% \small
% \bibitem[\protect\citeauthoryear{{Salpeter}}{{Salpeter}}{1955}]
% {SalpeterE_55a}{Salpeter} E.~E., 1955, ApJ, 121, 161 
% \end{thebibliography}

% The Scientific Justification should provide a clear explanation of the background, context and motivation for your proposed observations. 
% It should justify the expected outcome(s) and why they are important and timely.

\clearpage           % Page separator - DO NOT EDIT OR REMOVE!

%%%%%%%%%%%%%%%%%%%%%%%%%%%%%%%%%%%%%%%%%%%%%%%%%%%%%%%%%%%%%%%%%%%%%%%%%%%%%%%%%%%%%%%%%%%%%%%%%%

% *** NOTE: ONE PAGE is to be used for the TECHNICAL JUSTIFICATION,
% *** FLEXIBILITY \& MINIMUM TIME, and TARGET LIST.  Long target lists
% *** can be included as an attached sheet at the end of the
% *** proposal. We recommend using at least 1/2 page for the TECHNICAL
% *** JUSTIFICATION


\noindent{\bf TECHNICAL JUSTIFICATION:}

% INSTRUCTIONS: The STACK cannot grant time that it cannot clearly see
% will result in the proposed science, so this section is crucial.
% This section should describe the feasibility of your science on Keck;
% it is NOT the place to discuss, e.g. your target selection, or any
% other aspect of scientific justification.  You must *DEMONSTRATE* and
% *JUSTIFY* (not just merely state) why and how your proposed
% observations can be undertaken with Keck.  You may also want to
% justify why your science requires Keck and/or its instruments, and
% not those available on smaller telescopes.

% All proposals MUST address the following points in the description:

% -  Describe why the chosen instrument is the best choice.

% -  Include all relevant instrument setup details (grating angles,
% filters, slit widths, CCD binning etc.).  

% -  State the number of requested nights and FULLY JUSTIFY the
% exposure times and overhead calculations for all targets and all
% set-ups/filters.  Including a worked-out example for a "typical
% target" from longer target lists is acceptable (but you must justify
% why the chosen target is "typical").  For a range of targets, include
% the values for extreme cases. 

% ** NOTE: the STACK must be able to reproduce these values without 
%    further assumptions **

% -  Detail the visibility of the targets for the requested nights.
% This must include the number of hours the field is above the Nasmyth
% platforms (rising or setting limits imposed by Keck I and Keck II).
% See http://www2.keck.hawaii.edu/inst/common/TelLimits.html for
% details.  Include the amount of time that they are *not* visible (if
% any) and detail the observations and science that will be made during
% this time.

% Note that you MUST include a back-up program in the BACKUP PROGRAM 
% section below.  The backup program must address all of the above 
% points, however, they can be to a lower level of detail.

% You must also fully *justify* the technical feasibility of any
% possible variations in the instrument, setup parameters etc. in the
% "Flexibility \& Minimum Time" section.


\noindent{\bf FLEXIBILITY \& MINIMUM TIME:} % - DO NOT EDIT OR REMOVE

% INSTRUCTIONS: Flexibility: Describe any aspect of your proposed
% observations which are flexible with respect to, for example,
% lunation, dates, instrument availability, instrument setting,
% distribution of time (e.g. multiple night gaps allowed) etc.  
% Flexibility will aid us in approving and scheduling your proposal.

% Minimum Time: Please also state the minimum amount of time in which 
% you can undertake your science.  Proposers should keep in mind that 
% SUT have a limited number of nights on Keck each semester and those 
% nights must be distributed over each telescope and over 
% bright/grey/dark time `fairly'.  


\clearpage           % Page separator - DO NOT EDIT OR REMOVE!


\noindent{\bf JUSTIFICATION FOR  DIRECTOR'S DISCRETIONARY TIME (IF APPLICABLE) :}  % - DO NOT EDIT OR REMOVE
% INSTRUCTIONS: If applicable discuss how their proposal meets the DDT proposal selection criteria (details in the CfP).} 


\bigskip



\noindent{\bf BACKUP PROGRAM:}     % - DO NOT EDIT OR REMOVE
\\ {\em A total of 2 pages maximum are allowed for the sections BACKUP PROGRAM through to
RESPONSE TO STACK FEEDBACK.}

% INSTRUCTIONS: Describe a backup program that you will undertake if
% prevailing weather conditions at the telescope, including seeing,
% prevent execution of your main program.  Also, if your proposal relies
% on instruments or facilities which are still experimental or regularly
% lose time to technical problems (e.g.$\sim$laser guide star adaptive
% optics) then specify your contingency plans.

% This section MUST include the science you will obtain with the BACKUP
% PROGRAM(S) and MUST included details as described in TECHNICAL
% JUSTIFICATION for the proposed science (albeit, detailed to a lesser
% extent).

\bigskip

\noindent{\bf CONTRACT END:}    % - DO NOT EDIT OR REMOVE

% INSTRUCTIONS: Permanent staff and staff and postdocs on fixed-term 
% contracts are eligible as PIs. PIs must clearly state their 
% contract end date.  
% As a reminder, postdocs must have a ongoing staff member as a Co-I.  
% This staff member need to be clearly identified here and 
% will be held responsible for the data and outcomes of the observations.

\bigskip

\noindent{\bf APPLICANTS' ROLES:}    % - DO NOT EDIT OR REMOVE

% INSTRUCTIONS: Describe in detail the role of the PI and each of the
% CoIs, particularly those not at Swinburne.  Also specify who the
% leading authors on papers arising from this proposal are likely to be.

\bigskip

\noindent{\bf OBSERVING and DATA REDUCTION EXPERIENCE:}  % - DO NOT EDIT OR REMOVE


% INSTRUCTIONS: Describe the PIs observing experience and how it relates
% to the proposed instrument.  If a PI is inexperienced with the
% proposed instrument, one must specify how the PI will obtain the
% necessary experience prior to the observing run.  This may occur in
% several ways: 
% 	(1) CoIs are experienced with the instrument and will
% 		aid the PI with observations. 
% 	(2) the PI may arrive in Waimea ahead of time to observe others use 
% 		the instrument and/or discuss the instrument with the instrument specialist 
% 	(3) the PI may eavesdrop on other CAS observers using the same, or a similar,
% 		instrument during the previous semester.
% We strongly recommend that a local expert is involved if the 
% PI doesn't have the relevant observational and data reduction
% experience.




\bigskip

\noindent{\bf STUDENT THESIS INVOLVEMENT:}  % - DO NOT EDIT OR REMOVE


% INSTRUCTIONS: If this proposal forms a part of a student CoI's thesis,
% describe the role it plays in their thesis work and how crucial it is
% to the completion of their thesis.
% The responsibility of the data remains with the PI or the delegated 
% staff member with an ongoing contract as stated in the Section CONTRACT END.

\bigskip



\noindent{\bf OTHER APPLICATIONS:} % - DO NOT EDIT OR REMOVE

% INSTRUCTIONS: Specify the details (including proposal title, time
% requested and/or already awarded, time allocation committee) for this
% project and related projects over the last year and in the coming
% semester which include the PI or any of the CoIs of this proposal.

\bigskip

\noindent{\bf FUTURE APPLICATIONS:} % - DO NOT EDIT OR REMOVE 

% INSTRUCTIONS: Justify the number of STACK Keck nights required in
% future semesters for the COMPLETION of the project described in this
% proposal.  If no additional nights are requested by the PI, the STACK
% will assume that the project will be complete after the current number
% requested is awarded.  If additional time is requested in the future
% for a completed project this request must be clearly explained (i.e.,
% new targets have been identified by the PI or in the literature) or it
% will be considered negatively in the ranking of the PIs proposals
% (unless awarded time was lost due to weather/instrument failure).

\bigskip

\noindent{\bf RESPONSE TO STACK FEEDBACK:}

% INSTRUCTIONS: This is an optional section to respond to STACK feedback on 
% a re-submitted proposal. 

\clearpage

\noindent{\bf TARGET LIST:}

% INSTRUCTIONS: Use the template table to include a full target list.
% If you run out of room for the target list, please attach an
% additional page at the end of the proposal that includes only the
% target list.

% \begin{table}[h]
% \begin{center}
% {\footnotesize\begin{tabular}{|c|c|c|c|c|c|c|}\hline
% Object ID & RA(h:m:s)  & Title (d:m:s) & mag. (band) & Redshift\\
% \hline 
% 3C 273    & 12:29:06.7 & $+$02:03:09.1 & 12.8 V      &  0.158  \\
% 3C 336    & 16:24:39.1 & $+$23:45:12.1 & 18.5 V      &  0.927  \\
% \hline 

% \end{tabular}}
% \end{center}
% \end{table}

\clearpage

\bigskip

\end{document}
