%%%%% This is a simple template for writing proposals for Keck
%%%%% Observing Support through the CAS SACK (Swinburne Advisory
%%%%% Committee for Keck).
%%%%%
%%%%% To produce a PDF from this file execute the following from a
%%%%% command line:
%%%%%
%%%%% latex STACK_proposal.tex ; latex STACK_proposal.tex ;
%%%%%   dvips STACK_proposal.dvi -o STACK_proposal.ps ;
%%%%%   ps2pdf14 -sPAPERSIZE=a4 STACK_proposal.ps STACK_proposal.pdf
%%%%% 
%%%%% or, if you prefer,
%%%%%
%%%%% pdflatex STACK_proposal.tex ; pdflatex STACK_proposal.tex
%%%%%
%%%%% Take note of the comments below starting with "%%%%%". DO NOT
%%%%% EDIT OR REMOVE commands with comments which tell you not to!

\documentclass[12pt,a4paper]{article} % - DO NOT EDIT OR REMOVE!
\usepackage{fancyhdr,txfonts} % - DO NOT EDIT OR REMOVE!

%%%%% Include your own packages here. For example, if you want use the
%%%%% natbib package, leave uncommented the first two commands below
%%%%% (the second command specifies puctuation within citations; the
%%%%% example shown gives something like the MNRAS style). The third
%%%%% commented-out command below includes the commonly used to
%%%%% include figures. The fourth commented-out command might be
%%%%% useful while you're drafting your proposal.
\usepackage{natbib}
\bibpunct{(}{)}{;}{a}{}{,}
%\usepackage{graphicx}
%\usepackage[light]{draftcopy}
\newcommand\doc{Keck Observatory Funding Request Form}
%%%%% Enter the Keck cover sheet code, PI surname and proposal title
%%%%% below. Please enter the Keck cover sheet code in the same format
%%%%% as given on the Keck cover sheet. NOTE: Include any formatting
%%%%% commands (e.g. boldface, \boldmath commands etc.) for your
%%%%% proposal title here.
\newcommand\keckcode{20???\_CS?????}
\newcommand\PIsurname{PI SURNAME}
\newcommand\proposaltitle{Title of proposal} 

% Page layout - DO NOT EDIT OR REMOVE!
\topmargin=-20mm
\headheight=0mm
\textwidth=176mm
\textheight=266mm
\oddsidemargin=-8mm
\evensidemargin=-8mm
\renewcommand\baselinestretch{1.0}
\pagestyle{fancyplain}
\lhead{\fancyplain{}{{\bf\MakeUppercase{\keckcode}}}}
\rhead{\fancyplain{}{\bf\thepage}}
\chead{\fancyplain{}{{\bf\MakeUppercase{\PIsurname}}}}
\lfoot{}

%%%%% Page layout - YOU MAY EDIT THESE IF YOU WISH.
%%%%%
%%%%% NOTE THAT IT IS STRONGLY RECOMMENDED NOT TO CROWD YOUR TEXT AND
%%%%% FORMATTING TOO MUCH BECAUSE IT MAKES READING THE PROPOSAL DIFFICULT.
%%%%% USE SIMPLE FORMATTING COMMANDS LIKE \smallskip, \bigskip, \vspace
%%%%% ETC. TO FORMAT YOUR TEXT.
\parindent=1.5em % Paragraph indentation length (in units of the size
                 % of an "M"). If you want to keep this in general but
                 % want a specific paragraph not to be indented, use
                 % the \noindent command at the start of that paragraph
\parskip=0.5em   % Space between paragraphs.

\begin{document}
\centerline{\Large{\underline{\bf \doc}}}\bigskip
\bigskip

% Proposal title - DO NOT EDIT OR REMOVE!
\centerline{\large{\bf \proposaltitle}}\bigskip

%%%%% Case for Keck Observing Support (KOS) funds.  This should be made by
%%%%% PIs of successful proposals once they have determined the amount of
%%%%% funding to be available from other sources, particularly the Faculty
%%%%% Research Committee and AMRFP funds. The PI needs to state the
%%%%% breakdown costs for each individal item (e.g., flight, car hire,
%%%%% accomindationes, etc.) along with the total costs and other sources of
%%%%% funds they have pursued.  KOS applications should be sent to the STACK
%%%%% technical secretary.  The technical secretary will check the reasons
%%%%% for the request and pass any queries to the SACK. The SACK will
%%%%% approve all allocations of KOS funds. The maximum KOS funding
%%%%% allocated to any given proposal will be \$2200. At the end of each
%%%%% calendar year residual funds in the Keck support budget for that year
%%%%% may be distributed retrospectively to that year's Keck users.  A TOTAL
%%%%% OF UP TO ONE PAGE is allowed.


\noindent{\bf CASE FOR OBSERVING SUPPORT FUNDS:}

\bigskip
Case for Keck Observing Support (KOS) funds.  This should be made by
PIs of successful proposals once they have determined the amount of
funding to be available from other sources, particularly the Faculty
Research Committee and AMRFP funds. The PI needs to state the
breakdown costs for each individal item (e.g., flight, car hire,
accomindationes, etc.) along with the total costs and other sources of
funds they have pursued.  KOS applications should be sent to the STACK
technical secretary.  The technical secretary will check the reasons
for the request and pass any queries to the SACK. The SACK will
approve all allocations of KOS funds. The maximum KOS funding
allocated to any given proposal will be \$2200. At the end of each
calendar year residual funds in the Keck support budget for that year
may be distributed retrospectively to that year's Keck users.  A TOTAL
OF UP TO ONE PAGE is allowed.

Example Table:
\bigskip
\begin{center}
\begin{tabular}{|r|r|r|}
\hline
{\bf Type of Expense}& {\bf Total  Expense} & {\bf KOS  Claim}\\
\hline
Air fare      &        &        \\
Car Rental    &        &        \\
Accommodation &        &        \\
Skybus        &        &         \\
Meals         &        &         \\

\hline
\bf{Total}    &        & \bf{\$2200.00}\\
\hline
\end{tabular}
\end{center}
\bigskip

% Signatures required once approved - DO NOT EDIT OR REMOVE!
\noindent{\large{\underline{ Signatures:}}}\smallskip

\noindent Swinburne PI (or representative):  \smallskip

\noindent Swinburne Advisory Committee for Keck Representive:  \smallskip

\noindent Date:

\end{document}
